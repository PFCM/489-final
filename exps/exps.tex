% !TEX root = ../proj_report_outline.tex

\chapter{Evaluation of architectures}\label{C:exps}
In this chapter we present experimental results comparing the performance of the proposed architectures.
\hl{something something something}.

\section{Synthetic Tasks}
We first evaluate the performance of the architectures on some synthetic tasks. These tasks are designed
to be difficult and to exercise various faculties of the model.

\subsection{Addition}
\subsubsection{Task}
This task is designed to test the networks' ability to store information for long time periods. It first
featured in \autocite{Hochreiter1997}, although we use the slightly different formulation found more 
recently in \autocite{Le2015}. The problem is a common benchmark for RNNs and has featured in a number of
recent works (\autocite{Arjovsky2015, Henaff2016, Barone2016, Neyshabur2016} for example).

The inputs for this task are sequences of length \(T\). There are two inputs, one sampled from a uniform
distribution over the range \([0,1]\) while the other is zero except for two locations where it is one.
The location of the first one is always chosen to be earlier than \(T/2\) while the second is after.
The goal is to output at the last time step the sum of the two random values that were presented when the
second input was one. Pseudocode for an algorithm to generate sequences is presented in the appendix,
section~\ref{sec:additionpseudo}.

\subsubsection{Experiment Setup}

\subsubsection{Results}

\subsection{Variable Binding}
\subsubsection{Task}

\subsubsection{Experiment Setup}

\subsubsection{Results}

\subsection{MNIST}
is really dumb

\section{Real-world Data}
Mostly testing rank as regulariser
\subsection{Polyphonic Music}
\subsection{PTB}
\subsection{War and Peace}