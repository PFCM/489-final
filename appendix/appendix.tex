% !TEX root = ../proj_report_outline.tex
\chapter{Additional Proofs}
\begin{prop} [Identity Tensor] \label{prop:identity}
	\(\tensor{H} \in \mathbb{R}^{N \times N \times N}\) such that
\begin{align}
\vec{x}^\mathsf{T}\tensor{H}\vec{y} = \vec{x} \odot \vec{y}, 
&&\forall \vec{x},\vec{y} \in \mathbb{R}^{N}
\end{align} implies
\begin{equation}
	H_{ijk} = \begin{cases}
		1 & \text{if}\;\;i = j = k \\
		0 & \text{otherwise.}
	\end{cases}
\end{equation}

\end{prop}
\begin{proof}
We prove briefly, by inspecting one component of the result. Let 
\(\vec{z} = \vec{x}^\mathsf{T}\tensor{H}\vec{y}\). Then
\begin{align}
	z_j &= \vec{x}^\mathsf{T}\mat{H}_{\midbullet j \midbullet}\vec{y} \\
		&= \sum_i^N\sum_k^N x_iH_{ijk}y_k
\end{align}
If \(z_j = x_jy_j\) as in the elementwise product, then it is clear we want \(H_{ijk}\) to be 
1 if \(i=j=k\). Further, if we ensure \(H_{ijk}\) is 0 when this is not the case we can see that
the rest of the terms in the sums will disappear.
\end{proof}