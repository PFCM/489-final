% !TEX root = ../proj_report_outline.tex

\chapter{Background and Related Work}\label{C:bg}
\section{Feed-Forward Neural Networks}
Feed-forward neural networks are a class of parameterised function approximators which consist of
artificial neurons arranged in layers. They arise from generalisations of the perceptron
\autocite{Rosenblatt1958},
indeed simple feed-forward networks are often termed `multi-layer perceptrons' (MLPs). MLPs solve
the key limitations of the early perceptron \autocite{Minsky1969}, but it was not until 1986
when a reliable and general purpose procedure (\emph{back-propagation}) was 
proposed \autocite{Rumelhart1986} to train them.

Conceptually it is possible to consider MLPs as a directed acyclic graph, where each node consists
of a `neuron' or `unit' with the direction of the edges indicating the way information flows from
the input of the function being modelled to the output. For an MLP this graph is divided into layers.
The first layer represents the input (with one unit per coordinate) while subsequent layers
represent individual neurons. In a standard fully-connected network, every node is connected to all
nodes in the subsequent layer. When an input is presented to the network, each neuron computes
an activation by a weighted sum over its inputs followed by the application of some non-linear
function. 
%Specifically a unit that receives inputs \((x_1, x_2, \ldots, x_n)\) will compute its
%activation \(h\) as:
%\begin{equation}\label{eq:perceptron}
%	h = f\left(\sum_{i=1}^n x_iw_i\right)
%\end{equation} where \((w_1, w_2, \ldots, w_n) + b\) are the weights the neuron gives its inputs and
%\(b\) is a bias, commonly added to enhance the expressive power.
%\(f : \mathbb{R} \to \mathbb{R}\) is some nonlinear function, common examples would be a threshold,
%the logistic sigmoid \(\sigma(x) = \frac{1}{1 + e^{-x}}\) or a linear rectifier
%\(\rho(x) = \max(0, x)\) \autocite{Nair}.
Computationally and
notationally it is efficient to express many neurons at once, using vectors and matrices.
If we consider the input to a network as a vector of features \(\vec{x}\),
and let \(\mat{W}\) be a matrix
whose rows contain the weights for each neuron, the output is computed by
%\begin{equation}\label{eq:vecperceptron}
%	h = f\left(\tran{\vec{w}}\vec{x} + b\right)
%\end{equation} where \(\vec{w}\) is the weights from eq. \eqref{eq:perceptron} as a vector.
%We can generalise this to compute an entire layer of activations at once. Let \(\mat{W}\) be a matrix
%whose rows contain the weights for each neuron and \(\vec{b}\) a vector
%containing all of their biases. We can then compute their
%activations at once with
\begin{equation} \label{eq:matperceptron}
	\vec{h} = f(\mat{W}\vec{x} + \vec{b})
\end{equation} where \(f: \mathbb{R}^n \mapsto \mathbb{R}^n\) is a non-linear function applied
element-wise. Common examples include the logistic sigmoid \(\sigma(x) = \frac{1}{1 + e^{-x}}\) or
a linear rectifier \(\rho(x) = \max(0, x)\) \autocite{Nair}. We also include a bias term \(\vec{b}\)
which is also learned, enhancing the range of the possible transformations greatly.

This gives us a single layer of neurons, a classic MLP will perform this process twice. 
The last layer
is often linear:
\begin{equation}
	\hat{\vec{y}} = \mat{W}^{(2)}f(\mat{W}^{(1)}\vec{x} + \vec{b}^{(1)}) + \vec{b}^{(2)}.
\end{equation} This can be though of as proceeding in stages -- first compute \(\vec{h}\) as before
and then using another set of weights compute a new transformation of it to get the final output.

Such neural networks are universal function approximators \autocite{Hornik1989}. However it is often
beneficial in practice to increase the depth as this can allow to express more complex relationships
with only a moderate increase in parameters to be learned \autocite{Telgarsky2016}. This is done
by recursively applying the above transform with a fresh set of weights and biases. Omitting the
biases for clarity this gives us the following for a network with \(l-1\) hidden layers:
\begin{equation}
	\hat{\vec{y}} = \mat{W}^{(l)}f(\mat{W}^{(l-1)}f(\dots f(\mat{W}^{(1)}\vec{x}))).
\end{equation} Naively adding depth in this fashion can cause the network to become very challenging
to train \autocite{Glorot2010, Saxe2013}. Several methods have been proposed which successfully
alleviate this, such as layer-wise pre-training \autocite{Hinton2006,Vincent2010} and more
recently residual or skip connections which change the structure of the graph to allow better flow of 
information \autocite{He2015, Huang2016}.

\subsection{Training}
These networks are typically trained in a supervised fashion. Denote the parameters of the network
\(\theta = \{\mat{W}^{(l)}, \ldots, \mat{W}^{(1)}, \vec{b}^{(l)}, \ldots, \vec{b}^{(l)}\}\) and
the output of the network for a given input \(\vec{x}\) and set of parameters as 
\(\hat{\vec{y}} = \mathcal{F}(\vec{x};\; \theta)\). Suppose we have a set of \(m\) data items
\(\mathcal{D} = \{(\vec{x}_i, \vec{y}_i)\}_{i=1}^m\) where each item is an input-output pair. 
Let \(\mathcal{X}\) be the set of possible \(\vec{x}\) values and \(\mathcal{Y}\) the set of
possible \(\vec{y}\) values.
We
wish to learn a mapping which will transform each  \(\vec{x}_i\) into \(\vec{y}_i\).
To do this we define a loss function
\(\mathcal{L} : \mathcal{Y} \times \mathcal{Y} \mapsto \mathbb{R}\) which maps pairs of valid outputs
to a real number, such that \(\mathcal{L}(\vec{y}, \hat{\vec{y}}) = 0\) implies 
\(\vec{y} = \hat{\vec{y}}\). Typical loss functions are the squared error
\( \label{eq:squarederror}
	\mathcal{L}_{SE}(\vec{y}, \hat{\vec{y}}) 
	%= ||\vec{y} - \hat{\vec{y}}||^2_2 
	= \sum_d (y_d - \hat{y}_d)^2
\) or the cross entropy
\(
	\mathcal{L}_{CE}(\vec{y}, \hat{\vec{y}}) = -\sum_d y_d \log(\hat{y}_d),
\) ensuring some safeguards are put in place to avoid attempting to compute \(\log 0\).
The cross entropy is typically preferred when the targets can be interpreted as
probabilities.\footnote{For example, discrete class labels -- we can convert them to a one-of-\(n\)
(often termed ``one-hot'') taking discrete labels between 1 and \(n\) to \(n\) dimensional vectors
with \(0\) in all positions bar one, which has its element set to \(1\). We can then view these
vectors as a (very certain) probability distribution over classes. In this case the cross entropy
corresponds exactly to the negative log-likelihood of the correct class.} This requires converting
the output of the network to a probability, typically done with either the sigmoid function 
mentioned above or the softmax \(s_i(\vec{x}) = e^{x_i}/\sum_j e^{x_j}\).

The best set of parameters for the network, \(\theta^*\), is the set which minimises the
loss across the data items. Training the network amounts to solving the following
optimisation problem:
\begin{equation}\label{eq:nnoptimise}
	\theta^* = 
	\argmin_{\theta} \sum_{i=1}^m \mathcal{L}(\vec{y_i}, \mathcal{F}(\vec{x}_i;\; \theta)).
\end{equation}

For neural networks this is typically done using variations on gradient descent. In its simplest
form, each weight is updated at each iteration as follows:
\begin{equation} \label{eq:graddescent}
	\mat{W} \gets \mat{W} - \eta \cdot \nabla_{\mat{W}}\mathcal{L}
\end{equation} where \(\eta\) is a hyper parameter adjusting the step size (called the
\emph{learning rate}) and \(\nabla_{\mat{W}}\mathcal{L}\) is the gradient of the loss with respect
to \(\mat{W}\). Typically the gradient is evaluated over a small subset or ``mini-batch''
 of the data per iteration which gives a noisy estimate of the true gradient.
The gradients of deep networks can be calculated using \emph{back-propagation}, which amounts
to the chain rule of calculus and the observation that neural networks consist of many
composed functions \autocite{Rumelhart1986}. In practice this is assisted greatly by software
packages which can determine the derivatives of a given function using automatic or symbolic
differentiation such as Tensorflow \autocite{Abadi2015} or Theano
 \autocite{TheTheanoDevelopmentTeam2016}.

\section{Recurrent Neural Networks}
Recurrent Neural Networks (RNNs) considered here generalise feed-forward networks to
address problems in which we wish to map a \emph{sequence} of inputs vectors
\(\vec{x} = (\vec{x}_1, \vec{x}_2, \dots \vec{x}_T) \) to a sequence of output vectors
\(\vec{y} = (\vec{y}_1, \vec{y}_2, \dots \vec{y}_T) \). They have been applied successful to a
wide range of tasks which can be framed in this way including statistical language modelling
\autocite{Mikolov2012} (including machine translation \autocite{Cho2014, Wu2016a}), speech 
recognition \autocite{Graves2006}, polyphonic music 
modelling \autocite{Boulanger-Lewandowski2012}, music classification \autocite{Choi2016},
image generation \autocite{Gregor2015}, automatic captioning \autocite{Vinyals2016, Xu2015}  and more.

\subsection{Classical Formulation}
An RNN is able to maintain context over a sequence by transferring its hidden state from one
time step to the next. We refer to the vector of states at time \(t\) as \(\vec{h}_t\).
The classic RNN (often termed ``Vanilla'') originally proposed in \autocite{Elman1990}
computes its hidden states with the following recurrence:
\begin{equation}
	\vec{h}_t = f(\mat{W}\vec{h}_{t-1} + \mat{U}\vec{x}_t +  \vec{b})
\label{eq:vanillarnn}
\end{equation} where \(f(\cdot)\) is some elementwise non-linearity, often the hyperbolic tangent:
\(\tanh(x) = \frac{e^{x} - e^{-x}}{e^{x} + e^{-x}}\). Final outputs are then computed from these
states in the same fashion as the feed-forward networks above. This remains the same for the
extended architectures considered below, the key difference is the manner in which the states
are produced.

Equation \eqref{eq:vanillarnn} is closely related to the building block of a 
feed-forward network. The key difference is the (square) matrix \(\mat{W}\) which contains weights
controlling how the previous state affects the computation of the new activations.

\subsection{Training}
We can train this (or any of the variants we will see subsequently) using back-propagation.
Often termed ``Back Propagation Through Time'' \autocite{Werbos1990}, this requires using the
chain rule to determine the gradients of the loss with respect to the network parameters in
the same manner as feed-forward networks.

To illustrate this, consider a loss function for the whole sequence of
the form
\begin{equation}
	\mathcal{L}(\hat{\vec{y}}_1, \hat{\vec{y}}_2, \dots, \hat{\vec{y}}_T,
				\vec{y}_1, \vec{y}_2, \dots, \vec{y}_T)
	= \sum_{i=1}^T \mathcal{L}_i(\hat{\vec{y}}_i, \vec{y}_i),
	\label{eq:seqloss}
\end{equation} which is a sum of the loss accrued at each time step. This captures all common
cases including sequence classification or regression, as the \(\mathcal{L}_i\) may simply return
\(0\) for all but the last time step. To find gradients of the loss with respect the parameters
which generate the hidden states, we must first find the gradient of the loss with respect to the
hidden states themselves. Choosing a hidden state \(i\) somewhere in the sequence we have:
\begin{equation}
	\nabla_{\vec{h}_i}\mathcal{L} = \sum_{j=i}^t \nabla_{\vec{h}_i}\mathcal{L}_j
	\label{eq:delhL}
\end{equation} from the definition of the loss and the fact that a hidden state may affect all
future losses. To determine each \(\nabla_{\vec{h_i}}\mathcal{L}_j\) (noting \(j \geq i\)), we
apply the chain rule, to back-propagate the error from time \(j\) to time \(i\). This is the step
from which the algorithm derives its name, and simply requires multiplying through adjacent
time steps. Let \(\vec{z}_k = \mat{W}\vec{h}_{t-1} + \mat{U}\vec{x}_t +  \vec{b}\) be the
pre-activation of the hidden states. Then
\begin{align}
	\nabla_{\vec{h}_i} \mathcal{L}_j &= 
	\left(\prod_{k=i+1}^j \nabla_{\vec{h}_{k-1}} \vec{h}_k\right)
	(\nabla_{\vec{h}_j}\mathcal{L}_j) \\
	&=  \left(\prod_{k=i+1}^j \nabla_{\vec{z}_k}f \cdot \mat{W}\right)
	(\nabla_{\vec{h}_j}\mathcal{L}_j).
	\label{eq:bptt-v}
\end{align} This has two key components: \(\nabla_{\vec{h}_j}\mathcal{L}_j\) quantifies the degree
to which the hidden states at time \(j\) affect the loss (computing this will most likely require
further back-propagation through one or more output layers) while the second term in
equation~\eqref{eq:bptt-v} measures how much the hidden state at time \(i\) affects the hidden
state at time \(j\).

We can now derive an update rule for the parameters by observing\footnote{For clarity, when we
consider the gradient of anything with respect to a matrix, such as \(\nabla_{\mat{W}}\mathcal{L}\)
we implicitly vectorise the matrix. More precise notation would be
\(\nabla_{\vect(\mat{W})}\mathcal{L}\), but this adds extra distraction into already unfortunately
 cluttered equations. For a definition of the vectorisation operation see 
section~\ref{sec:tensorinterps}. This means \(\nabla_{\mat{W}}\mathcal{L}\) has shape \(nn \times 1\),
where \(n\) is the number of hidden states. Correspondingly, \(\nabla_{\mat{W}}\vec{h_j}\) has
shape \(nn \times n\), so the shapes of the matrix multiplication in eq.~\eqref{eq:vanillaupdate}
are appropriately defined.
}
\begin{align}
	\nabla_{\mat{W}} \mathcal{L} &= \sum_{i=1}^T \nabla_{\mat{W}} \mathcal{L}_i \\
	&= \sum_{i=1}^T\sum_{j=1}^i(\nabla_{\mat{W}} \vec{h}_j)(\nabla_{\mat{h}_j} \mathcal{L}_i )
	\label{eq:vanillaupdate}
\end{align} 
and applying the above. For the input matrix \(\mat{U}\) and the bias the process is the same.

\subsection{Issues with Vanilla RNNs}
\subsubsection{Vanishing and Exploding Gradients}
Equation~\eqref{eq:bptt-v} reveals a key pathology of the vanilla RNN -- vanishing gradients.
This occurs when the gradient of the loss vanishes to a negligibly small value as we
propagate it backward in time, leading to a negligible update to the weights.
 \(\nabla_{\vec{h}_j}\mathcal{L}_j\) (a column vector) is multiplied by a long product of
 matrices, alternating between \(\nabla_{\vec{z}_k} f\) and \(\mat{W}\). If we assume for
 illustrative purposes that \(f(\cdot)\) is the identity function (so we have a linear network),
then the loss vector is multiplied by \(\mat{W}\) taken to the \((j-i)\)-th power. If the largest
eigenvalue of \(\mat{W}\) is large, then this will cause the gradient to rapidly explode.
If the largest eigenvalue is small, then the gradient will vanish. This issue was first presented
in 1994 \autocite{Bengio1994} where it was shown to be unavoidable if the network needs to both store
information and be robust to noise. However, this has not precluded further research into alleviating
the main symptoms, for a thorough treatment including necessary conditions 
for vanishing and the complementary exploding problem, see \autocite{Pascanu2012}.

In the non-linear case, this remains a serious issue. While exploding gradients are often
mitigated by using a \emph{saturating} non-linearity so that the gradient tends to zero as the
hidden states grow, this only exacerbates the vanishing problem. 

The key symptom of vanishing gradients in RNNs is that it makes it much harder to learn to
store information for long time periods \autocite{Bengio1994}. This makes RNNs struggle to
solve simple-seeming tasks in which the solution requires remembering an input for many
time steps.

\subsubsection{Butterfly Effect}
A second issue when training RNNs can be illustrated by viewing them as iterated non-linear	
dynamical systems and thus susceptible to the ``butterfly effect'': seemingly negligible changes
in initial conditions can lead to catastrophic changes after a number of iterations
\autocite{Lorenz1963}. In RNNs this manifests as near-discontinuity of the loss surface
\autocite{Pascanu2012} as a change (for example, to a weight during back-propagation) which may
be in a downward direction on the loss surface can cause instabilities which lead to steep
increases in loss. This problem is not as well studied as vanishing gradients although some
partial solutions exist such as clipping the norm of gradients \autocite{Pascanu2012} or
using a regulariser to encourage gradual changes in hidden state \autocite{Krueger2016}.

\section{Popular Alternatives: LSTM and GRU}
To address these fundamental problems a number of alternate architectures have been proposed.
Here we will outline two popular variants: the Long Short Term Memory (LSTM) and the Gated
Recurrent Unit (GRU). Both of these belong to a class of \emph{gated} RNNs, which have a
markedly different method of computing a new state.

The LSTM was proposed to alleviate the vanishing gradient problem. It uses a number of gates
to control the flow of information during the computation of new states. Although a number
of subtle variants exist, the standard LSTM we will consider here has the form: 
\autocite{Hochreiter1997, Graves2013}
\begin{align}\label{eq:LSTM}
	\vec{h}_t &= \vec{o}_i \odot \tau(\vec{c}_t)\\
	\vec{c}_t &= \vec{f}_t \odot \vec{c}_{t-1} + \vec{i}_t \odot \vec{g}_t\\
	\vec{g}_t &= \tau(\mat{W}_g \vec{c}_{t-1} + \mat{U}_g \vec{x}_t + \vec{b}_g)\\
	\vec{o}_t &= \sigma(\mat{W}_o \vec{c}_{t-1} + \mat{U}_o \vec{x}_t + \vec{b}_o)\\
	\vec{f}_t &= \sigma(\mat{W}_f \vec{c}_{t-1} + \mat{U}_f \vec{x}_t + \vec{b}_f)\\
	\vec{i}_t &= \sigma(\mat{W}_i \vec{c}_{t-1} + \mat{U}_i \vec{x}_t + \vec{b}_i)
\end{align} where \(\tau(\cdot)\) refers to the elementwise \(\tanh\), \(\sigma(\cdot)\) the
elementwise logistic sigmoid \(\sigma(x) = \frac{1}{1 + e^{-x}}\) and \(\cdot \odot \cdot\)
is used to denote elementwise multiplication between vectors. These equations can be hard to take
at face value -- the key elements are \(\vec{i}_t, \vec{o}_t, \vec{f}_t\), termed the
\emph{input}, \emph{output} and \emph{forget} gates respectively, are computed in the same fashion
as the activations of a vanilla neural network but use sigmoid activation function which varies
smoothly between zero and one. Combining this with element-wise multiplication has the eponymous
gating effect, attenuating the contributions of other components. 

The output gate is fairly straightforward, it simply allows the network to prevent its hidden
state from being exposed. The forget and input gates have a more difficult role to characterise.
Taken together, these control the acceptance or rejection of new information by modulating the
amount by which the new candidate state \(\vec{g}_t\) is accepted into the hidden state
\(\vec{c}_t\).

A closely related architecture proposed much more recently by Cho et al. \autocite{Cho2014} is the
GRU, which computes its state as follows:
\begin{align}\label{eq:GRU}
	\vec{h}_t &= \vec{f}_t \odot \vec{h}_{t-1} + (1 - \vec{f}_t)\odot\vec{z}_t\\
	\vec{z}_t &= \tau(\mat{W}_z(\vec{r}_t \odot \vec{h}_{t-1}) + \mat{U}_z\vec{x}_t + \vec{b}_z)\\
	\vec{f}_t &= \sigma(\mat{W}_f\vec{h}_{t-1} + \mat{U}_f\vec{x}_t + \vec{b}_f)\\
	\vec{r}_t &= \sigma(\mat{W}_r\vec{h}_{t-1} + \mat{U}_r\vec{x}_t + \vec{b}_r).
\end{align} This is a slightly simpler form than the LSTM although the alterations go beyond
simply removing the output gate. Notably, the forget gate now controls both parts of the state
update. Further, in the computation of \(\vec{z}_t\) there is a departure from the vanilla
RNN-style building block that makes up all of the LSTM's operations. This is interesting as a
half of the model's parameters, and therefore a large part of its computational power, is
dedicated towards computing state updates. However, the mechanism of their computation places
significant emphasis on using temporally local information to do so -- the \emph{reset} gate
\(\vec{r}_t\) provides the model with the ability to ignore parts of its state.

The key shared component of these architectures is an \emph{additive} state update. Another way
of phrasing this is that while the vanilla RNN attempts to learn an opaque function
\(\vec{h}_t = \mathcal{F}(\vec{x}_t, \vec{h}_{t-1})\), these gated architectures instead learn
a \emph{residual} mapping \(\vec{h}_t = \mathcal{F}(\vec{x}_t, \vec{h}_{t-1}) + \vec{h}_{t-1}\).
By itself, this would alleviate the main cause of vanishing gradients 
\autocite{Jozefowicz2015, Hochreiter1997}. This is equivalent to adding a skip connection,
allowing the state to skip a time step and is directly analogous to the residual connections
now commonly used to address the vanishing gradient problem in very deep feed-forward networks
\autocite{He2015, Duvenaud2014, Szegedy2016}. Unfortunately the presence of the gate complicates
this perspective -- this will be analysed in detail in chapter~\ref{C:arch}.

These architectures are often used at large scale to solve very hard real world tasks. However,
the synthetic tasks on which the LSTM was originally validated \autocite{Hochreiter1997}
 overlook a key requirement of a successful RNN architecture. The tasks focus on problems involving long
time dependencies, but all involve remember either symbols from a fixed, small vocabulary or a single
real number. We suggest that it is also necessary to evaluate the ability of RNNs to learn to use
all of their memory cells at once to store distributed representations of the current context from which
elements of the sequence can be reconstructed. While this ability is used implicitly, especially in
architectural advancements targeted at language modelling \autocite{Cho2014b, Chan2015, Luong2016},
no specific empirical validation of this ability is regularly carried out.


\section{Other Ideas}
While the LSTM and the GRU are by far the most popular RNN architectures, many other solutions
have been proposed to the problems that plague the Vanilla RNN. We divide these into two categories:
training and memory augmentation. Training solutions focus on adjusting the structure of the RNN or
the training algorithm to avoid vanishing or exploding gradients and hence allow it to learn to solve
difficult tasks. Memory augmentation approaches view an RNN as a device which interacts with a
memory (in the form of its hidden states) and either augment the memory with additional
structures or impose specific access patterns to encourage more appropriate use of this memory.

\subsection{Enhancing Training}
Methods of addressing issues when training RNNs can be further divided into two categories:
architectural and algorithmic.
Architectural solutions, such as the LSTM and GRU, seek to alter
the way in which hidden states are calculated. Algorithmic solutions look to change the training
process itself to better avoid or handle the various pathologies. These are complementary lines of
enquiry -- advanced architectures can still benefit from more sophisticated training techniques.

\subsubsection{Architectural Solutions}
An early attempt to solve the vanishing gradient problem involves feeding inputs to different units
at different
rates. The primary motivation for this seems to be to force the network to attend to longer
time-scales by passing some parts of it information at a slower rate, although the authors note an
intuition that more abstract representations of the sequence will change slower and hence not
require re-calculation at every step \autocite{Hihi1995}. This idea has been revisited recently in
the form of the Clockwork RNN \autocite{Koutnik2014} which flattens the network, simply forcing
the various hidden states to be updated at different predefined rates. Further approaches in this
direction attempt to have the network learn the time scale at which it should update -- recent
examples include Hierarchical Multiscale RNNs \autocite{Chung2016} and Gated Feedback RNNs
\autocite{Chung2015} which suggest various ways of connecting, via additional gated connections,
the outputs of various layers in a stacked RNN architecture.

These approaches have been successful, although it is not precisely clear how adding extra
multiplicative gates could alleviate the vanishing gradient problem. Secondly, many of these
approaches add significant extra structures to the network (typically a large LSTM)
to force it to behave in a manner which
it was already capable of. That significant extra structures need to be added to able to reliably
train it to express such behaviours seems to suggest that a more fundamental change in the model
would be wise.

A closely related approach is augmenting RNNs to add an attention mechanism. This is especially
popular in translation tasks, in which a common approach is to use two separate RNNs in an
\emph{encoder-decoder} arrangement where the first network encodes the sentence into a fixed length
vector and the second expands the vector into a new sentence in a different language
\autocite{Cho2014b, Luong2016}. Content-based attention is often used to improve these models, 
by allowing the decoder access to all previous states of the encoder via a weighted sum. Weights
are derived from similarities between each encoder state and a vector produced by the decoder at
each step \autocite{Chan2015, Bahdanau2015}.

While this achieves excellent performance at
extraordinary scale \autocite{Wu2016a}, it also requires computing similarities for every step of the
output with every step of the input, which will scale quadratically in sentence length. Consequently,
it seems like a useful avenue of research to allow the encoder RNN to learn a more useful reduced
representation of the sentence to remove the need for the attention mechanism.

A simpler recent architecture of note is the Unitary Evolution RNN
\autocite{Arjovsky2015} which guarantee the eigenvalues of the recurrent weight matrix have a
magnitude of one. While this leads to provably non-vanishing or exploding gradients, in
practice they still seem to struggle to learn to store information for very long time periods.
This is potentially due to the seemingly ad-hoc composition of unitary operators used to allow
unconstrained optimisation of parameters while maintaining the desired properties of the
recurrent matrix.

Another principled way to approach the design of RNNs is to use \emph{strongly typed} architectures.
\autocite{Balduzzi2016} This approach is inspired by the desire to alter the main source of problems
in classic RNNs -- the recurrent weight matrix. The argument is that continually applying the same
weight matrix to the hidden states leads to incoherent features in the hidden state. The solution is
to attempt to \emph{diagonalise} the state updates. This corresponds to ensure all operations on the
hidden state apply elementwise, and that the input should be the only item projected through weight
matrices.

Another line of enquiry which can help with long time-dependencies focuses on the initialisation
of the network. IRNNs \autocite{Le2015}, which simply initialise the recurrent weights matrix to the
 identity (presaged by the nearly diagonal initialisation in \autocite{Mikolov2015}), perform
remarkably well on pathological tasks. The importance of the initialisation of the recurrent weights
matrix was further emphasised in \autocite{Henaff2016} where marked differences were found between
initialising a slightly modified vanilla RNN with the identity matrix or a random orthogonal matrix.
While simply initialising the network to have certain beneficial properties provides no guarantees
on final performance after training, it is important to note the significant results reported
especially as they are likely to at least be partially transferable to any given architecture.

\subsubsection{Algorithmic Solutions}
The second class of attempts to address the issue focuses on techniques for training the network.
 The first of these is to use approximate
second-order methods to try and rescale the gradients appropriately. A number of notable attempts
used Hessian-Free Optimization \autocite{Martens2011, Boulanger-Lewandowski2012} and achieved
remarkable results. Perhaps more interestingly, it was subsequently found that many of the results
could in fact be achieved with very careful tuning of standard stochastic gradient descent with
momentum and careful initialisation \autocite{Sutskever2013a}.

Another interesting approach which seems to help learn networks learn tasks requiring longer time
dependencies is to add a regularisation term into the loss to penalise the difference between
successive hidden states \autocite{Krueger2016}. This encourages the network to learn smooth
transitions and may help it ``bootstrap'' itself past the vanishing gradients.

A notable recent work applies the recently proposed path-SGD \autocite{Neyshabur2015} to Vanilla
RNNs with ReLU activations. This involves re-casting the RNN as a very deep feed-forward network with
shared weights across times-steps. The authors then regularise the update based on the scale of
the individual paths through the network. This allows them to achieve significant results with an
architecture that is otherwise very challenging to train -- the linear rectifier is unbounded in
one direction which typically leads rapidly to exploding gradients \autocite{Neyshabur2016}.


\subsection{Memory}
This line of research attempts to enhance the capabilities of RNNs by focusing on the way they
interact with their memory. There are two key motivations for this kind of work. 
One aim is to decouple the memory from the network so that it can store and interact with arbitrary
quantities of information. The second is related to vanishing gradients in the hope that a novel
method of storing or addressing information will be more robust to long time delays.

A notable effort to decouple RNNs from their memory comes in Neural Turing Machine proposed in
\autocite{Graves2014}. This uses a neural network (which may or may not be recurrent itself) to
attend to a large matrix of memory. Addressing for both reading and writing is done via a
address mechanism mostly based on the softmax. It was evaluated on a number of synthetic tasks
which mostly involved accumulation such as remembering a sequence of arbitrary patterns and outputting
them after a number of time steps in a particular order.

Subsequently, Grefenstette et al. propose a set of neural Deques, Queues and Stacks which attempt to
address the limitations of the fixed size memory of the Neural Turing Machine 
\autocite{Grefenstette2015}. A similar approach is also found in Joulin and Mikolov's Stack Augmented
RNN \autocite{Joulin2015}. These allow for unbounded memory with efficient access, at the cost of
the random access nature of the memory. Primarily these are proposed as augmentations to an RNN,
with the idea being that a standard RNN could operate as a small working memory while the more
complex data structure (with more awkward access semantics) can be used for long-term, more exact
storage. This is an exciting direction as it enables something akin to a learnable Von Neumann
architecture \autocite{Graves2014}.

Other such approaches have also arisen such as the Neural Random-Access Machine, which is similar to
a Neural Turing Machine except that it solves a number of the inefficiencies with a clever addressing
mechanism \autocite{Kurach2016}. Other related works include Pointer Networks \autocite{Vinyals2015}
and Neural GPUs \autocite{Kaiser2015}. These tend to involve adding fairly complex structures on
top of an LSTM. While they are highly successful on synthetic tasks when compared to an
un-augmented LSTM, very few of the tasks seem like the kind of task that an RNN should inherently be
incapable of solving. Nearly all of the tasks addressed require a fixed size memory, and solutions
can be imagined composed of carefully writing to it and reading from it. Hence it seems as though it
is worth revisiting the basic architecture, rather than searching for distinct modules to add on.

An approach in this direction is Danihelka et al.'s Associative LSTM \autocite{Danihelka2016}. 
This work incorporates an associative memory model into the LSTM by adjusting several of the
operations to reflect something close to a Holographic Reduced Representation \autocite{Plate1995}.
This is a very encouraging angle -- to re-think the way in which the RNN uses its memory at a basic
level seems like the most efficient and sensible way to address the issues. The results were
very promising on some interesting synthetic tasks, focusing on the ability of the network to do
key-value retrieval which should be a strength of the Holographic Reduced Representation.
Unfortunately simply adjusting the network to sue such a reduced representation was insufficient for
good performance and a number of complicating factors had to be incorporated, including keeping many
redundant copies of memory which diminishes somewhat the elegance of the solution.

Many of these ideas seem to be founded on an intuition which has not been clearly formalised. In
general, the task of an RNN is to learn a mapping from one sequence to another sequence one step at
a time using an intermediate memory (the hidden states). Intuitively a solution must consist of
writing to and reading from memory, both conditioned on input. In an RNN, reading from the memory is
trivially attended to as typically the entire memory is used to produce the output at each time
step. However, we would suggest identifying two separate types of writing to memory: accumulating
and forgetting. Many of the architectures discussed below (and the GRU and LSTM above) have separate
mechanisms for these two tasks. 
We consider an accumulation to be an updating of the state, typically additively, to incorporate
new information. Forgetting is simply wiping clean a section of the memory -- this tends to be
achieved multiplicatively. Different tasks will require a different balance of the two. Instinctively,
a classification tasks in which the network is only evaluated on its final output would favour
more accumulative solutions. Conversely, tasks which may not even have a fixed length but rather
require constant prediction should be better suited to networks with an ability to forget information
that is no longer relevant.

\section{Tensors in Neural Networks}
The final set of related works provide the key motivation for the next section. In general, many of
the above architectures contain multiplicative gates. Multiplicative gating structures are not novel,
introduced by Hinton as early as 1981 \autocite{Hinton1981}. Sigaud et al. separate modern uses into
two categories based on the intuition behind the application \autocite{Sigaud2015}. The first class
seeks to use the gate in a manner akin to a transistor in a digital circuit, seeking to switch the
flow of information between computational units on or off. This captures the way such connections are
used in LSTMs as well as their feed-forward cousins Highway Networks \autocite{Srivastava2015}. The
second class of gated networks aims to implement a multiplicative connection between inputs. The
latter class strictly generalises the former. Intuitively, the former class captures those models
where two inputs to a computational unit are multiplied before they are processed, while the second
admits some transformations before the multiplication is realised. Two excellent reviews can be
found in \autocite{Memisevic2011, Sigaud2015} and rather than repeat in detail the history we
attempt to bring out the most salient examples which make clear the intuitions behind incorporating
multiplicative interactions. Finally, we mention a few particularly very recent applications
that have achieved significant successes after publication of the aforementioned reviews.

The earliest examples of multiplicative interactions in the context of the current Deep Learning
movement is in the Gated Boltzmann machine \autocite{Memisevic2007} which generalised Restricted
Boltzmann Machines \autocite{Smolensky1986} (a class of probabilistic graphical models very closely
related to neural networks) to model conditional relationships between pairs of images. These 
parameterise a multiplicative connection with a three-way tensor \(\tensor{W}\).
The likelihood of a hidden state
being \(1\) given a pair of inputs \(\vec{x}\) and \(\vec{y}\) is defined as
\begin{equation}\label{eq:grbm}
	p(h_k | \vec{x},\vec{y}) = \sigma\left(\sum_{ij} W_{ijk}x_iy_j \right)
\end{equation} where \(\sigma(\cdot)\) is the sigmoid discussed earlier \autocite{Memisevic2007}.
This architecture very naturally extends the operation of a feed-forward network to deal with
two inputs, but the size of the
tensor \(\tensor{W}\) is a serious limitation. This work outlines three key points: 
multiplicative interactions are somehow a natural way of handling multiple distinct inputs,
multiplicative interactions are properly parameterised by tensors
and tensors can be prohibitively large.
This last point is addressed in later works in the same field by factorising the tensor into three
matrices \autocite{Taylor, Memisevic2010}.

The Multiplicative RNN \autocite{Martens2011a, Sutskever2013} picks up both the notion that
multiplicative interactions are sensible ways to model interactions between two inputs and the
factorisation from the above. Applied in a language modelling context, the reasoning is that it
makes sense to allow the current input symbol to affect the recurrent transition matrix. These MRNNs
achieved significant results, especially when combined with Hessian Free \autocite{Martens2011},
a method of approximating second order optimisation.

Very recent methods have revisited a number of these ideas, although often with little to no mention
of the underlying tensorial structure. Of note is the Multiplicative Integration RNN which proposes
adjusting the building block of eq.~\eqref{eq:vanillarnn} to use element-wise multiplication instead
of addition \autocite{Wu2016}.
 This corresponds to factorising the tensor into two matrices and has an extreme gating
effect, especially during back-propagation.
In order to successfully learn, it was found necessary to add additional bias terms to the
point where the final model required four weight matrices and five bias vectors per unit. In some
experiments they incorporate this approach into an LSTM, leading to a model which would have nearly
\(16\) times the parameters of the simplest RNN proposed by eq.~\eqref{eq:vanillarnn}. We suggest
that paying more attention to the tensorial structure of the desired interaction would lead to more
thoroughly grounded models, and propose one such model in chapter~\ref{C:arch}.

LSTM-style gating also continues to receive attention, being employed in Gated-Attention Readers
to combine representations of questions and potential answers \autocite{Dhingra2016}. It is also used
in PixelCNN and WaveNet \autocite{VandenOord2016, Oord2016}, allowing to model to choose how
particular features propagate upwards through layers.

It is important to realise that all of these multiplicative structures have an inherent tensor
structure, whether it is made implicit or not. In chapter~\ref{C:tens} we investigate with more
rigour the types of tensor products which generalise many of these multiplicative interactions.


